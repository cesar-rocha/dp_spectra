\documentclass[11pt]{article}

%% WRY has commented out some unused packages %%
%% If needed, activate these by uncommenting
\usepackage{geometry}                % See geometry.pdf to learn the layout options. There are lots.
%\geometry{letterpaper}                   % ... or a4paper or a5paper or ... 
\geometry{a4paper,left=2.5cm,right=2.5cm,top=2.5cm,bottom=2.5cm}
%\geometry{landscape}                % Activate for rotated page geometry
%\usepackage[parfill]{parskip}    % Activate to begin paragraphs with an empty line rather than an indent

\usepackage{natbib}

%for figures
%\usepackage{graphicx}

\usepackage{color}
\definecolor{mygreen}{RGB}{28,172,0} % color values Red, Green, Blue
\definecolor{mylilas}{RGB}{170,55,241}
%% for graphics this one is also OK:
\usepackage{epsfig}

%% AMS mathsymbols are enabled with
\usepackage{amssymb,amsmath}

%% more options in enumerate
\usepackage{enumerate}
%\usepackage{enumitem}

%% insert code
\usepackage{listings}

\usepackage[utf8]{inputenc}

\usepackage{hyperref}

%% To make really wide whats that cover everything:
\usepackage{scalerel}
\usepackage{stackengine}
\stackMath
\def\hatgap{2pt}
\def\subdown{-2pt}
\newcommand\what[2][]{%
\renewcommand\stackalignment{l}%
\stackon[\hatgap]{#2}{%
\stretchto{%
    \scalerel*[\widthof{$#2$}]{\kern-.6pt\bigwedge\kern-.6pt}%
    {\rule[-\textheight/2]{1ex}{\textheight}}%WIDTH-LIMITED BIG WEDGE
}{0.5ex}% THIS SQUEEZES THE WEDGE TO 0.5ex HEIGHT
_{\smash{\belowbaseline[\subdown]{\scriptstyle#1}}}%
}}

% Default fixed font does not support bold face
\DeclareFixedFont{\ttb}{T1}{txtt}{bx}{n}{12} % for bold
\DeclareFixedFont{\ttm}{T1}{txtt}{m}{n}{12}  % for normal

% Custom colors
\usepackage{color}
\definecolor{deepblue}{rgb}{0,0,0.5}
\definecolor{deepred}{rgb}{0.6,0,0}
\definecolor{deepgreen}{rgb}{0,0.5,0}

% colors
\usepackage{graphicx,xcolor,lipsum}


\usepackage{mathtools}

\usepackage{graphicx}
\newcommand*{\matminus}{%
  \leavevmode
  \hphantom{0}%
  \llap{%
    \settowidth{\dimen0 }{$0$}%
    \resizebox{1.1\dimen0 }{\height}{$-$}%
  }%
}

% commmand for this doc
\newcommand{\So}{\mathcal{S}}
\newcommand{\bvth}{\bar{\vth}}
\newcommand{\kappat}{\kappa_{tot}}
\newcommand{\kappae}{\kappa_{e}}
\newcommand{\kappaN}{\kappa_{n}}
\newcommand{\kappaoc}{\kappa_{oc}}


\title{Notes on Modal Projection and non-zero vertical shear at the boundary, etc.}
\author{Cesar \& Bill}
\date{\today}

\begin{document}

\include{symbols}


\maketitle

%\section{}

We want to calculate the mean quasigeostrophic potential vorticity gradients:
\beq
    \label{meanQGPV}
    Q_y = \beta - \sL V(z)\com \qquad \text{and} \qquad Q_x = \sL U(z) \com
\eeq
where $\sL \defn \p_z (f_0/N)^2 \p_z $ is the stretching operator. Calculation of $\sL U$ and $\sL V$
from data is typically noisy. To reduce the noise, we want to project the basic state velocity onto
the familiar vertical modes $\sp_n$ that satisfy
\beq
\label{modes}
\sL \sp_n = - \kappa_n^2 \sp_n\com
\eeq
with $\sp_n' = 0\com~ z = -h, 0$. However, with nonzero vertical shear at the boundaries we cannot differentiate the series for $U(z)$ and $V(z) $. So, instead of the tradition series, we write
\beq
\label{decomp}
U(z) \approx U_s(z) + U_i(z)\com
\eeq
where $U_i(z)$ is an ``interior'', boundary shearless, velocity component:
\beq
\label{Uint}
U_i(z) = \sum_{n=0}^{\nmax} \breve{U}_n \sp_n\com \qquad \breve{U}_n = \tfrac{1}{h}\int_{-h}^{0}\!\! \sp_n \, [U(z)-U_s(z)]  \dd z\com
\eeq
and $U_s(z)$ satisfies $U'(\zm) = U_s'(\zm)$ and  $U'(\zp) = U_s'(\zp)$. Also in \eqref{decomp} the approximate sign stems from the truncated nature of the series \eqref{Uint}. A simple choice is
\beq
\label{dUsurf}
\frac{\dd U_s}{\dd z} = \frac{N^2(z)}{N^2(0)} \frac{z + h}{h} U'(0) - \frac{N^2(z)}{N^2(-h)} \frac{z}{h} U'(-h)\per
\eeq
The reason we include the $N^2$ factors above, e.g. $N^2(z)/N^2(0)$ is that we want to calculate $\sL U_s$ as smoothly as possible, and hence independently of derivatives of $N^2(z)$. With this choice we have
\beq
\label{Usurf}
U_s(z) = \frac{U'(0)}{N^2(0) h}\int_{0}^{z} N^2(z') (z' + h) \dd z' -  \frac{U'(-h)}{N^2(-h) h}\int_{0}^{z} N^2(z') z'  \dd z  + A \com 
\eeq
where $A$ is determined by imposing a no net transport condition
\beq
\int_{-h}^{0} U_s'(z) \dd z = 0\per
\eeq
We obtain
\beq
\label{A}
A =  -\frac{U'(0)}{N^2(0) h^2}\int_{-h}^0 \dd z \int_{0}^{z} N^2(z') (z' + h) \dd z' +  \frac{U'(-h)}{N^2(-h) h^2} \int_{-h}^0 \dd z  \int_{0}^{z} N^2(z') z'  \dd z
\eeq
The integrals above are calculated numerically for an arbitrary buoyancy profile $N^2(z)$. 

% for constant N(z) we calculate the integrals analytically
%\beq
%\label{A}
%A = \frac{1}{3 h }\left( \frac{U'(-h)}{2} - U'(0) \right) \per
%\eeq
%Thus
%\beq
%\label{Usurf2}
%U_s(z) = \frac{U'(0)}{h}\left( \frac{z^2}{2} + z h - \frac{1}{3}\right) - \frac{U'(-h)}{h} \left(\frac{z^2}{2} + \frac{1}{6} \right) \per
%\eeq


In summary, knowledge of the shear at the boundaries determines the ``surface'' velocity component $U_s(z)$ that
can be then be subtracted from the total velocity to obtain a ``interior'' velocity component $U_i(z)$. Because $U_i(z)$ is shearless at the boundaries, we can accurately calculate $\sL U$
\beq
\sL U \approx \sL U_s + \sL U_i  = \frac{f_0^2}{N^2(0)} \frac{U'(0)}{h}
- \frac{f_0^2}{N^2(-h)} \frac{U'(-h)}{h} - \sum_{n=1}^{\nmax} \kappa_n^2 \breve{U}_n \sp_n \per
\eeq

\end{document}
